\documentclass{article}                      
\usepackage{siunitx}                          \usepackage{setspace}                        
\usepackage{gensymb}
\usepackage{xcolor}                           \usepackage{caption}                          %\usepackage{subcaption}                      %\doublespacing
\singlespacing
\usepackage[none]{hyphenat}                   \usepackage{amssymb}
%\usepackage{relsize}
\usepackage[cmex10]{amsmath}                  \usepackage{mathtools}                        \usepackage{amsmath}
\usepackage{commath}
%\usepackage{amsthm}
%\interdisplaylinepenalty=2500                %\savesymbol{iint}                            %\usepackage{txfonts}
%\restoresymbol{TXF}{iint}
%\usepackage{wasysym}                         \usepackage{amsthm}                           \usepackage{mathrsfs}
\usepackage{txfonts}
\let\vec\mathbf{}
%\usepackage{stfloats}
\usepackage{float}
\usepackage{cite}
\usepackage{cases}
\usepackage{subfig}                           %\usepackage{xtab}                            \usepackage{longtable}
\usepackage{multirow}
%\usepackage{algorithm}
\usepackage{amssymb}                          %\usepackage{algpseudocode}
\usepackage{enumitem}
\usepackage{mathtools}
%\usepackage{eenrc}
%\usepackage[framemethod=tikz]{mdframed}      \usepackage{listings}                         \usepackage{listings}                         \usepackage[latin1]{inputenc}                 %% \usepackage{color}
\usepackage{titling}
%\usepackage{fulbigskip}
\usepackage{tikz}                             \usepackage{graphicx}
\begin{document}
\title{GATE QUES - CS 2002\\SOLUTION}
\date{}
\maketitle
\section{QUES}
\begin{enumerate}
	\item Consider the following logic circuit where inputs are functions f1,f2,f3 and output is ?:
      \tikzset{every picture/.style={line width=0.75pt}} %set default line width to 0.75pt        
\begin{tikzpicture}[x=0.75pt,y=0.75pt,yscale=-1,xscale=1, baseline=(XXXX.south) ]
\path (0,197);\path (434.4000244140625,0);\draw    ($(current bounding box.center)+(0,0.3em)$) node [anchor=south] (XXXX) {};
%Shape: Nand Gate [id:dp2890993872019729] 
\draw   (89.04,43.6) -- (104.71,43.6) .. controls (113.36,43.6) and (120.38,52.34) .. (120.38,63.1) .. controls (120.38,73.86) and (113.36,82.6) .. (104.71,82.6) -- (89.04,82.6) -- (89.04,43.6) -- cycle (78.6,50.1) -- (89.04,50.1) (78.6,76.1) -- (89.04,76.1) (126.64,63.1) -- (135,63.1) (120.38,63.1) .. controls (120.38,60.95) and (121.78,59.2) .. (123.51,59.2) .. controls (125.24,59.2) and (126.64,60.95) .. (126.64,63.1) .. controls (126.64,65.25) and (125.24,67) .. (123.51,67) .. controls (121.78,67) and (120.38,65.25) .. (120.38,63.1) -- cycle ;
%Shape: Not/Inverter Gate [id:dp19946932703366493] 
\draw   (167.81,77) -- (212.26,107) -- (167.81,137) -- (167.81,77) -- cycle (153,107) -- (167.81,107) (221.15,107) -- (233,107) (212.26,107) .. controls (212.26,103.69) and (214.25,101) .. (216.7,101) .. controls (219.16,101) and (221.15,103.69) .. (221.15,107) .. controls (221.15,110.31) and (219.16,113) .. (216.7,113) .. controls (214.25,113) and (212.26,110.31) .. (212.26,107) -- cycle ;
%Shape: Nand Gate [id:dp5606961917286739] 
\draw   (330.86,56.77) -- (353.09,56.77) .. controls (365.35,56.77) and (375.31,70.22) .. (375.31,86.77) .. controls (375.31,103.33) and (365.35,116.77) .. (353.09,116.77) -- (330.86,116.77) -- (330.86,56.77) -- cycle (316.05,66.77) -- (330.86,66.77) (316.05,106.77) -- (330.86,106.77) (384.2,86.77) -- (396.05,86.77) (375.31,86.77) .. controls (375.31,83.46) and (377.3,80.77) .. (379.75,80.77) .. controls (382.21,80.77) and (384.2,83.46) .. (384.2,86.77) .. controls (384.2,90.09) and (382.21,92.77) .. (379.75,92.77) .. controls (377.3,92.77) and (375.31,90.09) .. (375.31,86.77) -- cycle ;
%Straight Lines [id:da09054539939918449] 
\draw    (135,63.1) -- (316.05,66.77) ;
%Straight Lines [id:da8208682343487839] 
\draw    (233,107) -- (316.05,106.77) ;
% Text Node
\draw (88,94.7) node [anchor=north west][inner sep=0.75pt]    {$f3( x\ y\ z)$};
% Text Node
\draw (12,39.7) node [anchor=north west][inner sep=0.75pt]    {$f1( x\ y\ z)$};
% Text Node
\draw (11,64.7) node [anchor=north west][inner sep=0.75pt]    {$f2( x\ y\ z)$};
\end{tikzpicture}
TRUTH - TABLE


\begin{table}[!h]
        \centering
        
\begin{tabular}{|p{0.25\textwidth}|p{0.25\textwidth}|p{0.25\textwidth}|p{0.25\textwidth}|}
\hline 
 x & y & z & y(output) \\
\hline 
 0 & 0 & 0 & 0 \\
\hline 
 0 & 0 & 1 & 0 \\
\hline 
 0 & 1 & 0 & 0 \\
\hline 
 0 & 1 & 1 & 0 \\
\hline 
 1 & 0 & 0 & 0 \\
\hline 
 1 & 0 & 1 & 0 \\
\hline 
 1 & 1 & 0 & 1 \\
\hline 
 1 & 1 & 1 & 0 \\
 \hline
\end{tabular}
        
        \end{table}
\tikzset{every picture/.style={line width=0.75pt}} %set default line width to 0.75pt        
\begin{tikzpicture}[x=0.75pt,y=0.75pt,yscale=-1,xscale=1, baseline=(XXXX.south) ]
\path (0,389);\path (606.8000183105469,0);\draw    ($(current bounding box.center)+(0,0.3em)$) node [anchor=south] (XXXX) {};
%Shape: Grid [id:dp9703746717393453] 
\draw  [draw opacity=0] (143.39,70.85) -- (331.6,69.93) -- (332.3,163.15) -- (144.09,164.07) -- cycle ; \draw   (143.39,70.85) -- (144.09,164.07)(189.88,70.63) -- (190.58,163.84)(236.37,70.4) -- (237.07,163.61)(282.86,70.17) -- (283.56,163.39)(329.35,69.95) -- (330.05,163.16) ; \draw   (143.39,70.85) -- (331.6,69.93)(143.74,117.34) -- (331.95,116.42)(144.09,163.83) -- (332.3,162.91) ; \draw    ;
%Straight Lines [id:da3246491340892026] 
\draw    (143.39,70.85) -- (99.02,32.6) ;
% Text Node
\draw (154.66,84.25) node [anchor=north west][inner sep=0.75pt]   [align=left] {0};
% Text Node
\draw (201.15,122.32) node [anchor=north west][inner sep=0.75pt]   [align=left] {0};
% Text Node
\draw (201.15,84.01) node [anchor=north west][inner sep=0.75pt]   [align=left] {0};
% Text Node
\draw (154.95,122.32) node [anchor=north west][inner sep=0.75pt]   [align=left] {0};
% Text Node
\draw (247.64,83.78) node [anchor=north west][inner sep=0.75pt]   [align=left] {0};
% Text Node
\draw (247.93,121.85) node [anchor=north west][inner sep=0.75pt]   [align=left] {0};
% Text Node
\draw (294.13,83.55) node [anchor=north west][inner sep=0.75pt]   [align=left] {0};
% Text Node
\draw (294.42,121.62) node [anchor=north west][inner sep=0.75pt]   [align=left] {1};
% Text Node
\draw (155.6,49.51) node [anchor=north west][inner sep=0.75pt]   [align=left] {00};
% Text Node
\draw (204.41,49.51) node [anchor=north west][inner sep=0.75pt]   [align=left] {01};
% Text Node
\draw (247.92,49.51) node [anchor=north west][inner sep=0.75pt]   [align=left] {11};
% Text Node
\draw (292.75,51.42) node [anchor=north west][inner sep=0.75pt]   [align=left] {10};
% Text Node
\draw (115.43,83.78) node [anchor=north west][inner sep=0.75pt]   [align=left] {0};
% Text Node
\draw (120.08,119.95) node [anchor=north west][inner sep=0.75pt]   [align=left] {1};
% Text Node
\draw (121.08,33.06) node [anchor=north west][inner sep=0.75pt]   [align=left] {AB};
% Text Node
\draw (99.56,52.32) node [anchor=north west][inner sep=0.75pt]   [align=left] {C};
% Text Node
\draw (65,179) node [anchor=north west][inner sep=0.75pt]   [align=left] {{\large The final expression is of output is Y = \ F(x,y,!z)}};
% Text Node
\draw (68,234) node [anchor=north west][inner sep=0.75pt]   [align=left] {{\large Logic for the code will be Y = X\&\&Y\&\&!Z}};
\end{tikzpicture}


		
\end{enumerate}
\end{document}


